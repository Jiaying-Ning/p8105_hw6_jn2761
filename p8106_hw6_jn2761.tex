\PassOptionsToPackage{unicode=true}{hyperref} % options for packages loaded elsewhere
\PassOptionsToPackage{hyphens}{url}
%
\documentclass[]{article}
\usepackage{lmodern}
\usepackage{amssymb,amsmath}
\usepackage{ifxetex,ifluatex}
\usepackage{fixltx2e} % provides \textsubscript
\ifnum 0\ifxetex 1\fi\ifluatex 1\fi=0 % if pdftex
  \usepackage[T1]{fontenc}
  \usepackage[utf8]{inputenc}
  \usepackage{textcomp} % provides euro and other symbols
\else % if luatex or xelatex
  \usepackage{unicode-math}
  \defaultfontfeatures{Ligatures=TeX,Scale=MatchLowercase}
\fi
% use upquote if available, for straight quotes in verbatim environments
\IfFileExists{upquote.sty}{\usepackage{upquote}}{}
% use microtype if available
\IfFileExists{microtype.sty}{%
\usepackage[]{microtype}
\UseMicrotypeSet[protrusion]{basicmath} % disable protrusion for tt fonts
}{}
\IfFileExists{parskip.sty}{%
\usepackage{parskip}
}{% else
\setlength{\parindent}{0pt}
\setlength{\parskip}{6pt plus 2pt minus 1pt}
}
\usepackage{hyperref}
\hypersetup{
            pdftitle={p8105\_hw6\_jn2761},
            pdfauthor={jiaying Ning},
            pdfborder={0 0 0},
            breaklinks=true}
\urlstyle{same}  % don't use monospace font for urls
\usepackage[margin=1in]{geometry}
\usepackage{color}
\usepackage{fancyvrb}
\newcommand{\VerbBar}{|}
\newcommand{\VERB}{\Verb[commandchars=\\\{\}]}
\DefineVerbatimEnvironment{Highlighting}{Verbatim}{commandchars=\\\{\}}
% Add ',fontsize=\small' for more characters per line
\usepackage{framed}
\definecolor{shadecolor}{RGB}{248,248,248}
\newenvironment{Shaded}{\begin{snugshade}}{\end{snugshade}}
\newcommand{\AlertTok}[1]{\textcolor[rgb]{0.94,0.16,0.16}{#1}}
\newcommand{\AnnotationTok}[1]{\textcolor[rgb]{0.56,0.35,0.01}{\textbf{\textit{#1}}}}
\newcommand{\AttributeTok}[1]{\textcolor[rgb]{0.77,0.63,0.00}{#1}}
\newcommand{\BaseNTok}[1]{\textcolor[rgb]{0.00,0.00,0.81}{#1}}
\newcommand{\BuiltInTok}[1]{#1}
\newcommand{\CharTok}[1]{\textcolor[rgb]{0.31,0.60,0.02}{#1}}
\newcommand{\CommentTok}[1]{\textcolor[rgb]{0.56,0.35,0.01}{\textit{#1}}}
\newcommand{\CommentVarTok}[1]{\textcolor[rgb]{0.56,0.35,0.01}{\textbf{\textit{#1}}}}
\newcommand{\ConstantTok}[1]{\textcolor[rgb]{0.00,0.00,0.00}{#1}}
\newcommand{\ControlFlowTok}[1]{\textcolor[rgb]{0.13,0.29,0.53}{\textbf{#1}}}
\newcommand{\DataTypeTok}[1]{\textcolor[rgb]{0.13,0.29,0.53}{#1}}
\newcommand{\DecValTok}[1]{\textcolor[rgb]{0.00,0.00,0.81}{#1}}
\newcommand{\DocumentationTok}[1]{\textcolor[rgb]{0.56,0.35,0.01}{\textbf{\textit{#1}}}}
\newcommand{\ErrorTok}[1]{\textcolor[rgb]{0.64,0.00,0.00}{\textbf{#1}}}
\newcommand{\ExtensionTok}[1]{#1}
\newcommand{\FloatTok}[1]{\textcolor[rgb]{0.00,0.00,0.81}{#1}}
\newcommand{\FunctionTok}[1]{\textcolor[rgb]{0.00,0.00,0.00}{#1}}
\newcommand{\ImportTok}[1]{#1}
\newcommand{\InformationTok}[1]{\textcolor[rgb]{0.56,0.35,0.01}{\textbf{\textit{#1}}}}
\newcommand{\KeywordTok}[1]{\textcolor[rgb]{0.13,0.29,0.53}{\textbf{#1}}}
\newcommand{\NormalTok}[1]{#1}
\newcommand{\OperatorTok}[1]{\textcolor[rgb]{0.81,0.36,0.00}{\textbf{#1}}}
\newcommand{\OtherTok}[1]{\textcolor[rgb]{0.56,0.35,0.01}{#1}}
\newcommand{\PreprocessorTok}[1]{\textcolor[rgb]{0.56,0.35,0.01}{\textit{#1}}}
\newcommand{\RegionMarkerTok}[1]{#1}
\newcommand{\SpecialCharTok}[1]{\textcolor[rgb]{0.00,0.00,0.00}{#1}}
\newcommand{\SpecialStringTok}[1]{\textcolor[rgb]{0.31,0.60,0.02}{#1}}
\newcommand{\StringTok}[1]{\textcolor[rgb]{0.31,0.60,0.02}{#1}}
\newcommand{\VariableTok}[1]{\textcolor[rgb]{0.00,0.00,0.00}{#1}}
\newcommand{\VerbatimStringTok}[1]{\textcolor[rgb]{0.31,0.60,0.02}{#1}}
\newcommand{\WarningTok}[1]{\textcolor[rgb]{0.56,0.35,0.01}{\textbf{\textit{#1}}}}
\usepackage{graphicx,grffile}
\makeatletter
\def\maxwidth{\ifdim\Gin@nat@width>\linewidth\linewidth\else\Gin@nat@width\fi}
\def\maxheight{\ifdim\Gin@nat@height>\textheight\textheight\else\Gin@nat@height\fi}
\makeatother
% Scale images if necessary, so that they will not overflow the page
% margins by default, and it is still possible to overwrite the defaults
% using explicit options in \includegraphics[width, height, ...]{}
\setkeys{Gin}{width=\maxwidth,height=\maxheight,keepaspectratio}
\setlength{\emergencystretch}{3em}  % prevent overfull lines
\providecommand{\tightlist}{%
  \setlength{\itemsep}{0pt}\setlength{\parskip}{0pt}}
\setcounter{secnumdepth}{0}
% Redefines (sub)paragraphs to behave more like sections
\ifx\paragraph\undefined\else
\let\oldparagraph\paragraph
\renewcommand{\paragraph}[1]{\oldparagraph{#1}\mbox{}}
\fi
\ifx\subparagraph\undefined\else
\let\oldsubparagraph\subparagraph
\renewcommand{\subparagraph}[1]{\oldsubparagraph{#1}\mbox{}}
\fi

% set default figure placement to htbp
\makeatletter
\def\fps@figure{htbp}
\makeatother


\title{p8105\_hw6\_jn2761}
\author{jiaying Ning}
\date{11/30/2020}

\begin{document}
\maketitle

load packages

\hypertarget{problem-1}{%
\subsection{Problem 1}\label{problem-1}}

\hypertarget{import-data}{%
\subsubsection{Import data}\label{import-data}}

\begin{Shaded}
\begin{Highlighting}[]
\NormalTok{birthweight_df =}\StringTok{ }\KeywordTok{read_csv}\NormalTok{(}\StringTok{"./data/birthweight.csv"}\NormalTok{) }\OperatorTok
\StringTok{  }\KeywordTok{na.omit}\NormalTok{() }
\end{Highlighting}
\end{Shaded}

\begin{verbatim}
## Parsed with column specification:
## cols(
##   .default = col_double()
## )
\end{verbatim}

\begin{verbatim}
## See spec(...) for full column specifications.
\end{verbatim}

From description of variable, we know that variable\texttt{babysex},
\texttt{frace},\texttt{malform},\texttt{mrace} should be factor

\begin{Shaded}
\begin{Highlighting}[]
\NormalTok{birthweight_df=}
\NormalTok{birthweight_df }\OperatorTok
\StringTok{   }\KeywordTok{mutate}\NormalTok{(}\DataTypeTok{babysex =} \KeywordTok{ifelse}\NormalTok{(babysex }\OperatorTok{==}\StringTok{ }\DecValTok{1}\NormalTok{, }\StringTok{"male"}\NormalTok{,}\StringTok{"female"}\NormalTok{),}
     \DataTypeTok{malform =} \KeywordTok{ifelse}\NormalTok{(malform }\OperatorTok{==}\StringTok{ }\DecValTok{0}\NormalTok{, }\StringTok{"absent"}\NormalTok{,}\StringTok{"present"}\NormalTok{),}
          \DataTypeTok{frace =} \KeywordTok{recode}\NormalTok{(frace, }\StringTok{`}\DataTypeTok{1}\StringTok{`}\NormalTok{ =}\StringTok{ "White"}\NormalTok{, }\StringTok{`}\DataTypeTok{2}\StringTok{`}\NormalTok{=}\StringTok{"Black"}\NormalTok{, }\StringTok{`}\DataTypeTok{3}\StringTok{`}\NormalTok{ =}\StringTok{ "Asian"}\NormalTok{, }\StringTok{`}\DataTypeTok{4}\StringTok{`}\NormalTok{ =}\StringTok{ "Puerto Rican"}\NormalTok{, }\StringTok{`}\DataTypeTok{8}\StringTok{`}\NormalTok{ =}\StringTok{ "Other"}\NormalTok{, }\StringTok{`}\DataTypeTok{9}\StringTok{`}\NormalTok{ =}\StringTok{ "Unknown"}\NormalTok{),}
          \DataTypeTok{mrace =} \KeywordTok{recode}\NormalTok{(mrace, }\StringTok{`}\DataTypeTok{1}\StringTok{`}\NormalTok{ =}\StringTok{ "White"}\NormalTok{, }\StringTok{`}\DataTypeTok{2}\StringTok{`}\NormalTok{=}\StringTok{"Black"}\NormalTok{, }\StringTok{`}\DataTypeTok{3}\StringTok{`}\NormalTok{ =}\StringTok{ "Asian"}\NormalTok{, }\StringTok{`}\DataTypeTok{4}\StringTok{`}\NormalTok{ =}\StringTok{ "Puerto Rican"}\NormalTok{, }\StringTok{`}\DataTypeTok{8}\StringTok{`}\NormalTok{ =}\StringTok{ "Other"}\NormalTok{))}
\end{Highlighting}
\end{Shaded}

\hypertarget{propose-a-regression-model-for-birthweight}{%
\subsubsection{Propose a regression model for
birthweight}\label{propose-a-regression-model-for-birthweight}}

In order to find the variable of interest, I did research on the
potential risk factor that influence newborn's birthwright, and here is
what I found:

\begin{itemize}
\tightlist
\item
  \emph{The present study revealed that} \textbf{\emph{maternal
  illiteracy, exposure to passive smoking, late child bearing, shorter
  inter-pregnancy interval, previous LBW baby, maternal weight, weight
  gain during pregnancy, PIH, high risk pregnancy and late antenatal
  registration}} \emph{were the risk factors significantly associated
  with the birth weight of a newborn(Metgud,2012)}
\item
  \href{https://www.ncbi.nlm.nih.gov/pmc/articles/PMC3390317/\#}{Click
  Here to see the original paper}
\end{itemize}

For the current homework, I will choose variable\texttt{gaweeks}(as
measurement for inter-pregnancy interval) and \texttt{momage}(as
measurement for late child bearing)as my predictor, and \texttt{bwt} as
my outcome.

\begin{Shaded}
\begin{Highlighting}[]
\NormalTok{birthweight_df }\OperatorTok
\StringTok{  }\KeywordTok{ggplot}\NormalTok{(}\KeywordTok{aes}\NormalTok{(}\DataTypeTok{x=}\NormalTok{gaweeks,}\DataTypeTok{y=}\NormalTok{bwt,}\DataTypeTok{color=}\NormalTok{momage))}\OperatorTok{+}
\StringTok{  }\KeywordTok{geom_point}\NormalTok{(}\DataTypeTok{alpha=}\FloatTok{0.5}\NormalTok{) }
\end{Highlighting}
\end{Shaded}

\includegraphics[width=0.9\linewidth]{p8106_hw6_jn2761_files/figure-latex/unnamed-chunk-3-1}

Fit in a linear model:

\begin{Shaded}
\begin{Highlighting}[]
\NormalTok{Lin_mod =}\StringTok{ }\KeywordTok{lm}\NormalTok{(bwt }\OperatorTok{~}\StringTok{ }\NormalTok{momage }\OperatorTok{+}\StringTok{ }\NormalTok{gaweeks, }\DataTypeTok{data =}\NormalTok{ birthweight_df)}
\NormalTok{broom}\OperatorTok{::}\KeywordTok{tidy}\NormalTok{(Lin_mod)}
\end{Highlighting}
\end{Shaded}

\begin{verbatim}
## # A tibble: 3 x 5
##   term        estimate std.error statistic   p.value
##   <chr>          <dbl>     <dbl>     <dbl>     <dbl>
## 1 (Intercept)    300.      92.4       3.25 1.17e-  3
## 2 momage          11.9      1.83      6.50 9.23e- 11
## 3 gaweeks         65.3      2.25     29.0  2.83e-169
\end{verbatim}

\hypertarget{plot-of-model-residuals-against-fitted-value}{%
\subsubsection{plot of model residuals against fitted
value}\label{plot-of-model-residuals-against-fitted-value}}

\begin{Shaded}
\begin{Highlighting}[]
\KeywordTok{library}\NormalTok{(patchwork)}

\NormalTok{momage_res=}
\NormalTok{modelr}\OperatorTok{::}\KeywordTok{add_residuals}\NormalTok{(birthweight_df, Lin_mod) }\OperatorTok
\StringTok{  }\KeywordTok{ggplot}\NormalTok{(}\KeywordTok{aes}\NormalTok{(}\DataTypeTok{x=}\NormalTok{momage,}\DataTypeTok{y=}\NormalTok{resid))}\OperatorTok{+}
\StringTok{   }\KeywordTok{geom_point}\NormalTok{() }

\NormalTok{gaweeks_res=}
\NormalTok{modelr}\OperatorTok{::}\KeywordTok{add_residuals}\NormalTok{(birthweight_df, Lin_mod) }\OperatorTok
\StringTok{  }\KeywordTok{ggplot}\NormalTok{(}\KeywordTok{aes}\NormalTok{(}\DataTypeTok{x=}\NormalTok{gaweeks,}\DataTypeTok{y=}\NormalTok{resid))}\OperatorTok{+}
\StringTok{   }\KeywordTok{geom_point}\NormalTok{() }

\NormalTok{momage_res}\OperatorTok{+}\NormalTok{gaweeks_res}
\end{Highlighting}
\end{Shaded}

\includegraphics[width=0.9\linewidth]{p8106_hw6_jn2761_files/figure-latex/unnamed-chunk-5-1}

\hypertarget{comparing-with-two-other-model}{%
\subsubsection{Comparing with two other
model}\label{comparing-with-two-other-model}}

\hypertarget{length-at-birth-and-gestational-age-as-predictors-main-effects-only}{%
\paragraph{1. length at birth and gestational age as predictors (main
effects
only)}\label{length-at-birth-and-gestational-age-as-predictors-main-effects-only}}

\begin{Shaded}
\begin{Highlighting}[]
\NormalTok{mod1 =}\StringTok{ }\KeywordTok{lm}\NormalTok{(bwt }\OperatorTok{~}\StringTok{ }\NormalTok{blength }\OperatorTok{+}\StringTok{ }\NormalTok{gaweeks, }\DataTypeTok{data =}\NormalTok{ birthweight_df)}
\end{Highlighting}
\end{Shaded}

\hypertarget{head-circumference-length-sex-and-all-interactions-including-the-three-way-interaction}{%
\paragraph{2. head circumference, length, sex, and all interactions
(including the three-way
interaction)}\label{head-circumference-length-sex-and-all-interactions-including-the-three-way-interaction}}

\begin{Shaded}
\begin{Highlighting}[]
\NormalTok{mod2 =}\StringTok{ }\KeywordTok{lm}\NormalTok{(bwt }\OperatorTok{~}\StringTok{ }\NormalTok{bhead }\OperatorTok{+}\StringTok{ }\NormalTok{blength }\OperatorTok{+}\StringTok{ }\NormalTok{babysex }\OperatorTok{+}\StringTok{ }\NormalTok{bhead}\OperatorTok{*}\NormalTok{blength }\OperatorTok{+}\StringTok{ }\NormalTok{bhead}\OperatorTok{*}\NormalTok{babysex }\OperatorTok{+}\StringTok{ }\NormalTok{blength}\OperatorTok{*}\NormalTok{babysex }\OperatorTok{+}\StringTok{ }\NormalTok{bhead}\OperatorTok{*}\NormalTok{blength}\OperatorTok{*}\NormalTok{babysex, }\DataTypeTok{data =}\NormalTok{ birthweight_df)}
\end{Highlighting}
\end{Shaded}

\hypertarget{compare}{%
\paragraph{Compare}\label{compare}}

\begin{Shaded}
\begin{Highlighting}[]
\NormalTok{cv_df=}
\StringTok{  }\KeywordTok{crossv_mc}\NormalTok{(birthweight_df,}\DecValTok{100}\NormalTok{) }\OperatorTok
\StringTok{  }\KeywordTok{mutate}\NormalTok{(}
    \DataTypeTok{train =} \KeywordTok{map}\NormalTok{(train, as_tibble),}
    \DataTypeTok{test =} \KeywordTok{map}\NormalTok{(test, as_tibble)) }\OperatorTok\StringTok{ }
\StringTok{  }\KeywordTok{mutate}\NormalTok{(}
    \DataTypeTok{mod0  =} \KeywordTok{map}\NormalTok{(train, }\OperatorTok{~}\KeywordTok{lm}\NormalTok{(bwt }\OperatorTok{~}\StringTok{ }\NormalTok{momage }\OperatorTok{+}\StringTok{ }\NormalTok{gaweeks, }\DataTypeTok{data =}\NormalTok{ .x)),}
    \DataTypeTok{mod1  =} \KeywordTok{map}\NormalTok{(train, }\OperatorTok{~}\KeywordTok{lm}\NormalTok{(bwt }\OperatorTok{~}\StringTok{ }\NormalTok{blength }\OperatorTok{+}\StringTok{ }\NormalTok{gaweeks, }\DataTypeTok{data =}\NormalTok{ .x)),}
    \DataTypeTok{mod2  =} \KeywordTok{map}\NormalTok{(train, }\OperatorTok{~}\KeywordTok{lm}\NormalTok{(bwt }\OperatorTok{~}\StringTok{ }\NormalTok{bhead }\OperatorTok{+}\StringTok{ }\NormalTok{blength }\OperatorTok{+}\StringTok{ }\NormalTok{babysex }\OperatorTok{+}\StringTok{ }\NormalTok{bhead}\OperatorTok{*}\NormalTok{blength }\OperatorTok{+}\StringTok{ }\NormalTok{bhead}\OperatorTok{*}\NormalTok{babysex }\OperatorTok{+}\StringTok{ }\NormalTok{blength}\OperatorTok{*}\NormalTok{babysex }\OperatorTok{+}\StringTok{ }\NormalTok{bhead}\OperatorTok{*}\NormalTok{blength}\OperatorTok{*}\NormalTok{babysex, }\DataTypeTok{data =}\NormalTok{ .x))) }\OperatorTok\StringTok{ }
\StringTok{  }\KeywordTok{mutate}\NormalTok{(}
    \DataTypeTok{rmse_mod0 =} \KeywordTok{map2_dbl}\NormalTok{(mod0, test, }\OperatorTok{~}\KeywordTok{rmse}\NormalTok{(}\DataTypeTok{model =}\NormalTok{ .x, }\DataTypeTok{data =}\NormalTok{ .y)),}
    \DataTypeTok{rmse_mod1 =} \KeywordTok{map2_dbl}\NormalTok{(mod1, test, }\OperatorTok{~}\KeywordTok{rmse}\NormalTok{(}\DataTypeTok{model =}\NormalTok{ .x, }\DataTypeTok{data =}\NormalTok{ .y)),}
    \DataTypeTok{rmse_mod2 =} \KeywordTok{map2_dbl}\NormalTok{(mod2, test, }\OperatorTok{~}\KeywordTok{rmse}\NormalTok{(}\DataTypeTok{model =}\NormalTok{ .x, }\DataTypeTok{data =}\NormalTok{ .y)))}
\end{Highlighting}
\end{Shaded}

\begin{Shaded}
\begin{Highlighting}[]
\NormalTok{cv_df }\OperatorTok\StringTok{ }
\StringTok{  }\KeywordTok{select}\NormalTok{(}\KeywordTok{starts_with}\NormalTok{(}\StringTok{"rmse"}\NormalTok{)) }\OperatorTok\StringTok{ }
\StringTok{  }\KeywordTok{pivot_longer}\NormalTok{(}
    \KeywordTok{everything}\NormalTok{(),}
    \DataTypeTok{names_to =} \StringTok{"model"}\NormalTok{, }
    \DataTypeTok{values_to =} \StringTok{"rmse"}\NormalTok{,}
    \DataTypeTok{names_prefix =} \StringTok{"rmse_"}\NormalTok{) }\OperatorTok\StringTok{ }
\StringTok{  }\KeywordTok{mutate}\NormalTok{(}\DataTypeTok{model =} \KeywordTok{fct_inorder}\NormalTok{(model)) }\OperatorTok\StringTok{ }
\StringTok{  }\KeywordTok{ggplot}\NormalTok{(}\KeywordTok{aes}\NormalTok{(}\DataTypeTok{x =}\NormalTok{ model, }\DataTypeTok{y =}\NormalTok{ rmse)) }\OperatorTok{+}\StringTok{ }\KeywordTok{geom_violin}\NormalTok{()}
\end{Highlighting}
\end{Shaded}

\includegraphics[width=0.9\linewidth]{p8106_hw6_jn2761_files/figure-latex/unnamed-chunk-9-1}

\begin{Shaded}
\begin{Highlighting}[]
\NormalTok{cv_df }\OperatorTok\StringTok{ }
\StringTok{  }\KeywordTok{select}\NormalTok{(}\KeywordTok{starts_with}\NormalTok{(}\StringTok{"rmse"}\NormalTok{)) }\OperatorTok\StringTok{ }
\StringTok{  }\KeywordTok{pivot_longer}\NormalTok{(}
    \KeywordTok{everything}\NormalTok{(),}
    \DataTypeTok{names_to =} \StringTok{"model"}\NormalTok{, }
    \DataTypeTok{values_to =} \StringTok{"rmse"}\NormalTok{,}
    \DataTypeTok{names_prefix =} \StringTok{"rmse_"}\NormalTok{) }\OperatorTok
\StringTok{  }\KeywordTok{group_by}\NormalTok{(model) }\OperatorTok
\StringTok{  }\KeywordTok{summarize}\NormalTok{(}\DataTypeTok{avg_rmse =} \KeywordTok{mean}\NormalTok{(rmse))}
\end{Highlighting}
\end{Shaded}

\begin{verbatim}
## `summarise()` ungrouping output (override with `.groups` argument)
\end{verbatim}

\begin{verbatim}
## # A tibble: 3 x 2
##   model avg_rmse
##   <chr>    <dbl>
## 1 mod0      464.
## 2 mod1      332.
## 3 mod2      288.
\end{verbatim}

In here we see that model 2 (head circumference, length, sex, and all
interactions) have the smallest root mean squared errors, thus provide
the best model for predicting birthweight

\hypertarget{problem-3}{%
\subsection{Problem 3}\label{problem-3}}

\hypertarget{import-data-1}{%
\subsubsection{Import data}\label{import-data-1}}

\begin{Shaded}
\begin{Highlighting}[]
\NormalTok{weather_df =}\StringTok{ }
\StringTok{  }\NormalTok{rnoaa}\OperatorTok{::}\KeywordTok{meteo_pull_monitors}\NormalTok{(}
    \KeywordTok{c}\NormalTok{(}\StringTok{"USW00094728"}\NormalTok{),}
    \DataTypeTok{var =} \KeywordTok{c}\NormalTok{(}\StringTok{"PRCP"}\NormalTok{, }\StringTok{"TMIN"}\NormalTok{, }\StringTok{"TMAX"}\NormalTok{), }
    \DataTypeTok{date_min =} \StringTok{"2017-01-01"}\NormalTok{,}
    \DataTypeTok{date_max =} \StringTok{"2017-12-31"}\NormalTok{) }\OperatorTok
\StringTok{  }\KeywordTok{mutate}\NormalTok{(}
    \DataTypeTok{name =} \KeywordTok{recode}\NormalTok{(id, }\DataTypeTok{USW00094728 =} \StringTok{"CentralPark_NY"}\NormalTok{),}
    \DataTypeTok{tmin =}\NormalTok{ tmin }\OperatorTok{/}\StringTok{ }\DecValTok{10}\NormalTok{,}
    \DataTypeTok{tmax =}\NormalTok{ tmax }\OperatorTok{/}\StringTok{ }\DecValTok{10}\NormalTok{) }\OperatorTok
\StringTok{  }\KeywordTok{select}\NormalTok{(name, id, }\KeywordTok{everything}\NormalTok{())}
\end{Highlighting}
\end{Shaded}

\begin{verbatim}
## Registered S3 method overwritten by 'hoardr':
##   method           from
##   print.cache_info httr
\end{verbatim}

\begin{verbatim}
## using cached file: /Users/jiayingning/Library/Caches/R/noaa_ghcnd/USW00094728.dly
\end{verbatim}

\begin{verbatim}
## date created (size, mb): 2020-10-08 12:25:57 (7.525)
\end{verbatim}

\begin{verbatim}
## file min/max dates: 1869-01-01 / 2020-10-31
\end{verbatim}

\hypertarget{bootstrapping}{%
\subsubsection{Bootstrapping}\label{bootstrapping}}

\begin{Shaded}
\begin{Highlighting}[]
\NormalTok{weather_bootstrap=}
\NormalTok{weather_df }\OperatorTok\StringTok{ }
\StringTok{  }\NormalTok{modelr}\OperatorTok{::}\KeywordTok{bootstrap}\NormalTok{(}\DataTypeTok{n =} \DecValTok{5000}\NormalTok{) }\OperatorTok\StringTok{ }
\StringTok{  }\KeywordTok{mutate}\NormalTok{(}
    \DataTypeTok{models =} \KeywordTok{map}\NormalTok{(strap, }\OperatorTok{~}\StringTok{ }\KeywordTok{lm}\NormalTok{(tmax }\OperatorTok{~}\StringTok{ }\NormalTok{tmin, }\DataTypeTok{data =}\NormalTok{ .x)))}
\end{Highlighting}
\end{Shaded}

\hypertarget{plot-distribution-of-estimate}{%
\paragraph{Plot distribution of
estimate}\label{plot-distribution-of-estimate}}

\begin{Shaded}
\begin{Highlighting}[]
\NormalTok{weather_bootstrap }\OperatorTok\StringTok{ }
\StringTok{  }\KeywordTok{mutate}\NormalTok{(}\DataTypeTok{results =} \KeywordTok{map}\NormalTok{(models, broom}\OperatorTok{::}\NormalTok{tidy)) }\OperatorTok\StringTok{ }
\StringTok{  }\KeywordTok{select}\NormalTok{(results) }\OperatorTok\StringTok{ }
\StringTok{  }\KeywordTok{unnest}\NormalTok{(results) }\OperatorTok\StringTok{ }
\StringTok{    }\KeywordTok{filter}\NormalTok{(term }\OperatorTok{==}\StringTok{ "tmin"}\NormalTok{) }\OperatorTok\StringTok{ }
\StringTok{  }\KeywordTok{ggplot}\NormalTok{(}\KeywordTok{aes}\NormalTok{(}\DataTypeTok{x =}\NormalTok{ estimate)) }\OperatorTok{+}\StringTok{ }\KeywordTok{geom_density}\NormalTok{()}
\end{Highlighting}
\end{Shaded}

\includegraphics[width=0.9\linewidth]{p8106_hw6_jn2761_files/figure-latex/unnamed-chunk-13-1}

This distribution of regression coefficient is relatively normal, which
means that bootstrap has succefully resample the dataset with
replacement and establish normality and satisified distributional
assumption.

\hypertarget{confidence-interval-for-r-2}{%
\paragraph{95\% confidence interval for r̂
2}\label{confidence-interval-for-r-2}}

\begin{Shaded}
\begin{Highlighting}[]
\NormalTok{weather_bootstrap}\OperatorTok
\StringTok{  }\KeywordTok{mutate}\NormalTok{(}
    \DataTypeTok{results =} \KeywordTok{map}\NormalTok{(models, broom}\OperatorTok{::}\NormalTok{glance)) }\OperatorTok\StringTok{ }
\StringTok{  }\KeywordTok{select}\NormalTok{(results) }\OperatorTok\StringTok{ }
\StringTok{  }\KeywordTok{unnest}\NormalTok{(results) }\OperatorTok
\StringTok{  }\KeywordTok{summarise}\NormalTok{(}
    \DataTypeTok{ci_lower=}\KeywordTok{quantile}\NormalTok{(r.squared,}\FloatTok{0.025}\NormalTok{),}
    \DataTypeTok{ci_upper=}\KeywordTok{quantile}\NormalTok{(r.squared,}\FloatTok{0.975}\NormalTok{)}
\NormalTok{  )}
\end{Highlighting}
\end{Shaded}

\begin{verbatim}
## # A tibble: 1 x 2
##   ci_lower ci_upper
##      <dbl>    <dbl>
## 1    0.894    0.927
\end{verbatim}

\hypertarget{confidence-interval-for-logux3b2-0ux3b2-1}{%
\paragraph{95\% confidence interval for log(β̂ 0∗β̂
1)}\label{confidence-interval-for-logux3b2-0ux3b2-1}}

\begin{Shaded}
\begin{Highlighting}[]
\NormalTok{weather_bootstrap}\OperatorTok
\StringTok{   }\KeywordTok{mutate}\NormalTok{(}
    \DataTypeTok{results =} \KeywordTok{map}\NormalTok{(models, broom}\OperatorTok{::}\NormalTok{tidy))}\OperatorTok\StringTok{ }
\StringTok{  }\KeywordTok{select}\NormalTok{(results,.id) }\OperatorTok\StringTok{ }
\StringTok{  }\KeywordTok{unnest}\NormalTok{(results)  }\OperatorTok
\StringTok{  }\KeywordTok{select}\NormalTok{(term,estimate,.id)}\OperatorTok
\StringTok{  }\KeywordTok{pivot_wider}\NormalTok{(}
  \DataTypeTok{names_from =} \StringTok{"term"}\NormalTok{,  }
  \DataTypeTok{values_from =} \StringTok{"estimate"}\NormalTok{)}\OperatorTok
\StringTok{  }\KeywordTok{unnest}\NormalTok{()}\OperatorTok
\StringTok{  }\KeywordTok{mutate}\NormalTok{( }\DataTypeTok{log =} \KeywordTok{log}\NormalTok{(}\StringTok{`}\DataTypeTok{(Intercept)}\StringTok{`}\OperatorTok{*}\NormalTok{tmin)) }\OperatorTok
\StringTok{  }\KeywordTok{summarise}\NormalTok{(}
    \DataTypeTok{ci_lower=}\KeywordTok{quantile}\NormalTok{(log,}\FloatTok{0.025}\NormalTok{),}
    \DataTypeTok{ci_upper=}\KeywordTok{quantile}\NormalTok{(log,}\FloatTok{0.975}\NormalTok{)}
\NormalTok{  )}
\end{Highlighting}
\end{Shaded}

\begin{verbatim}
## Warning: `cols` is now required when using unnest().
## Please use `cols = c()`
\end{verbatim}

\begin{verbatim}
## # A tibble: 1 x 2
##   ci_lower ci_upper
##      <dbl>    <dbl>
## 1     1.96     2.06
\end{verbatim}

\end{document}
